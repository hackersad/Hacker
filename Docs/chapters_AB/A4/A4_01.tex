\section{Kapitel 9-10}

\subsection*{Aufgabe 1}

Beim Roulette gibt es im so genannten Kessel kleine F�cher mit den ganzen Zahlen zwischen 0 und 36. Eine Kugel wird nun in den Kessel geworfen; sie f�llt in genau eines dieser F�cher. 
Wir nehmen an, dass die Kugel bei einem Spiel mit gleicher Wahrscheinlichkeit in jedes der 37 F�cher f�llt. Setzt man nun 100 Euro auf die Zahl 36, so verliert man diesen Einsatz,wenn eine andere Zahl f�llt; man erh�lt den Einsatz zur�ck und zus�tzlich 3500 Euro Gewinn, wenn die Zahl 36 f�llt. Man k�nnte auch 100 Euro auf eine gerade Zahl setzen; in diesem Fall verliert man den Einsatz, wenn eine ungerade Zahl oder die 0 f�llt, man erh�lt den Einsatz zur�ck und zus�tzlich 100 Euro, wenn eine geradeZahl gr��er gleich 2 f�llt. Die Zufallsvariablen X und Y sollen den Gewinn beschreiben, den man bei einem Spiel der ersten bzw. zweiten oben genannten Variante macht (Verlust ist dabei negativer Gewinn). Geben Sie die Wahrscheinlichkeitsfunktionen der beiden Zufallsvariablen an und berechnen Sie ihre Erwartungswerte und Varianzen.