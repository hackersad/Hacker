\subsection*{Aufgabe 10}
�ber einen bin�ren Kommunikationskanal werden die Signale \glqq 0 \grqq und \glqq 1\grqq �bertragen. Aufgrund eines Rauschens werden die Signale manchmal verf�lscht, d.h. eine gesendete \glqq 0\grqq wird manchmal als \glqq 1 \grqq empfangen und umgekehrt. Eine \glqq 0 \grqq wird mit einer Wahrscheinlichkeit von 0,94, eine \glqq 1 \grqq mit einer Wahrscheinlichkeit
von 0,91 korrekt �bertragen. 55\% der gesendeten Zeichen sind eine \glqq 1\grqq?.
\begin{enumerate} [leftmargin=0.7cm, label=\alph*)]
\item Mit welcher Wahrscheinlichkeit wird bei �bertragung eines Zeichens eine \glqq 0\grqq empfangen? Mit welcher Wahrscheinlichkeit wird eine \glqq 1\grqq empfangen?
\item Bestimmen Sie die Wahrscheinlichkeit daf�r, dass eine \glqq 1\grqq gesendet wurde, wenn eine \glqq 0\grqq empfangen wurde.
\item Welcher Anteil der empfangenen Zeichen wurde anders verschickt als empfangen?
\end{enumerate}
L�sen die Aufgabe auf zwei Arten: Mit Hilfe der Wahrscheinlichkeitstheorie und alternativ durch Aufstellen einer Kreuztabelle.