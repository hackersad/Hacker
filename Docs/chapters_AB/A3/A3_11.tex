\subsection*{Aufgabe 11(Geburtstagsparadoxon)}
F�r das Folgende nehmen wir an, dass es 365 verschiedene Geburtstage gibt (d.h. wir ignorieren den 29. Februar), die alle gleich h�ufig sind (stimmt auch nicht exakt). Au�erdem nehmen wir an, dass Geburtstage verschiedener Personen voneinander unabh�ngig sind (d.h. wir ignorieren z.B. Zwillingsgeburten oder gemeinsame Geburtstagsfeiern).
\begin{enumerate} [leftmargin=0.7cm, label=\alph*)]
\item Nehmen Sie an, Sie sind zu einer Party eingeladen, an der noch 10 andere Personen teilnehmen. Mit welcher Wahrscheinlichkeit hat mindestens eine der anderen Personen am selben Tag Geburtstag wie Sie? Wie viele Personen m�ssten an der Party teilnehmen, damit diese Wahrscheinlichkeit gr��er als 50\% ist?
\item Mit welcher Wahrscheinlichkeit gibt es mindestens zwei Personen unter den 11 Teilnehmern der Party, die am gleichen Tag Geburtstag haben (egal wann)?
\item Geben Sie einen Tipp ab: Wie viele Personen m�ssen an der Party teilnehmen, damit die Wahrscheinlichkeit aus b) gr��er als 50\% ist? Das Ergebnis wird in der Vorlesung berechnet. Weil das Ergebnis (vermutlich) viel kleiner als Ihr Tipp ist, nennt man diese Aufgabe ein Paradoxon.
\end{enumerate}
