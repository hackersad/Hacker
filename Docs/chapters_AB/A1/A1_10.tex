\subsection*{Aufgabe 10}
Die so genannte Lebenserwartung einer Personengruppe ist (vereinfacht) das arithmetische
Mittel der (prognostizierten) vollendeten Lebensjahre zum Sterbezeitpunkt der einzelnen
Personen (Sterbealter in ganzen Jahren). Nach einem Zeitungsartikel lag 1871 die
Lebenserwartung eines Neugeborenen (im damaligen Deutschen Reich) bei 38 Jahren; 230
von 1000 S�uglingen starben dabei schon w�hrend des ersten Lebensjahres. Im Jahr 2011 lag
die Lebenserwartung eines Neugeborenen in Deutschland bei 80 Jahren; nur 4 von 1000
S�uglingen starben w�hrend des ersten Lebensjahres. Wie hat sich die Lebenserwartung von
Personen, die das erste Lebensjahr �berlebt haben, von 1871 bis 2011 ver�ndert?
