\section{Aufgabe E3}

\begin{enumerate}[leftmargin=1cm, label=\alph*)]
	\item F�nf medizinische Untersuchungsinstitute gaben als erforderlichen Zeitaufwand f�r die Bestimmung der Blutbestandteile die folgenden Zeitdauern je Blutprobe in Minuten an:
	
	\begin{center}
	\begin{tabular}{l|ccccc}
	Institut & 1 & 2 & 3 & 4 & 5\\
	\hline
	Dauer & 8 & 1 & 2 & 3 & 6\\
	\end{tabular}
	\end{center}
	
	\begin{enumerate}[label=\arabic*)]
		\item Berechnen Sie den mittleren Zeitaufwand, falls jedes Institut gleich viele Untersuchungen vornimmt.
		\vspace{4cm}
		\item Berechnen Sie den mittleren Zeitaufwand f�r die Bestimmung der Blutbestandteile,
		falls alle Institute �ber denselben Zeitraum von 8 Stunden ununterbrochen Blutproben
		untersuchen.
		\vspace{4cm}
	\end{enumerate}
	
	\item Man beobachtet das Wachstum von Schnittlauch, der zu Beginn des Beobachtungszeitraumes 5 cm hoch ist. Nach einer Woche ist er um 150\% gewachsen. W�hrend der zweiten Woche w�chst er um 20\% und w�hrend der dritten Woche um 45\%.Berechnen Sie das
	durchschnittliche Wachstum der Pflanze pro Woche.
\end{enumerate}