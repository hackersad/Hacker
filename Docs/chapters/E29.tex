\section{Aufgabe E29}
Ein Kartenspiel bestehe aus sechs blauen und einer gelben Karte. Aus diesem Stapel wird wiederholt eine
Karte zuf�llig gezogen. Ist die gezogene Karte gelb, wird sie in den Stapel zur�ckgelegt; ist sie blau, wird
sie beiseite gelegt und im Stapel durch eine neue gelbe Karte ersetzt. In jedem Fall wird der Kartenstapel
vor jeder neuen Ziehung gut gemischt.
\newline Eine Spielbank bietet folgendes Spiel an: Nach einem Einsatz von $k$ \euro \hspace{3pt}wird nach obiger Regel dreimal
gezogen. Ein Spieler erh�lt von der Bank
\begin{tabbing}
\hspace{4cm}\= \hspace{1.2cm}\= \hspace{0.8cm}\= \hspace{2.2cm}\= \hspace{1.8cm}\= \hspace{1.2cm}\= \kill
\> \EUR{100} \>f�r \>drei \>gezogene \>gelbe \> Karten,\\
\> \EUR{5} \>f�r \>genau zwei \>gezogene \>gelbe \> Karten,\\
\> \EUR{2} \>f�r \>genau eine \>gezogene \>gelbe \> Karte.\\
\end{tabbing}
Wird keine gelbe Karte gezogen, so ist der Einsatz f�r den Spieler verloren.
Wie gro� muss die Bank den Einsatz $k$ (in ganzen \euro) mindestens bemessen, um langfristig pro Spiel mindestens
\EUR{1} Gewinn zu machen?\vspace{10pt}
\newline \textit{Hinweis:} Sei $X$ die Anzahl gezogener gelber Karten und $Y$ der Auszahlungsbetrag.

